
\section{Setting up the API}

\begin{frame}

In this lab we will introduce the use of ERA5 and seasonal forecasting
data.  There are several key ways to access the datasets:

\begin{itemize}
\item Using the C3S online python-based toolbox to retrieve data and make simple
  analysis/plots (Advantage: less download volume. Disadvantage:
  system is slow, limited in type of analysis one can perform and can
  only access ERA5).
\item Use the web interface to select data, and then manually download
  using the browser (Advantage: simple to use, Disadvantage: Very
  cumbersome if you want to have more than a single variable)
\item Use the python API to download data directly to your desktop
  (Advantage, very flexible and convenient. Disadvantage, needs a
  little time to set up and you need anyway to refer to webpage to get
  example scripts)
\item{itemize}

Most regular users will use the API to access data, referring to the
webpage interface to help them set up the scripts.
\end{frame}

\begin{frame}
To set up the API, you need to 
\begin{itemize}
\item register with the C3S system 
\item Install the CDS API key (just once)
\item Install the CDS API client (just once)
\item Log in the C3S webpage to accept the terms and conditions on the
  dataset you wish to access (just once) and obtain an example script
\end{itemize} 

There is a very simple to follow online blog \url{available here}{
  https://cds.climate.copernicus.eu/api-how-to}.  Please follow the
steps carefully.

NOTE: If you use pip to install the cdsapi you will run the retrieval
script using python (python2). I recommend that you install instead
(or as well) using pip3, which will give you the opportunity to run
the script with python3 (python2 will 

NOTE: If you see this text 

\texttt{key: {uid}:{api-key}}

rather than a long random string of letters it means that \alert{you
  are not logged in!} Click on the ``login/register'' red button on
the top right to log in and resolve this. 

\end{frame}

\begin{frame}
Once 

\end{frame}
\mode<all>
